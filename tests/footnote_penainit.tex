\documentclass{article}
% 何が起きても必ず \pltx@foot@penalty は脚注の処理後は
% ゼロに初期化されなければならない
\textwidth20zw
\parindent0zw
\makeatletter
\def\PENALTYINFO{\the\pltx@foot@penalty}
\makeatother
\begin{document}

最初:\PENALTYINFO

あああああああああああああああ
あああああ(\footnotetext[1]{いいいいい}うううううううう\par

ここでもゼロ:\PENALTYINFO

あああああああああああああああ
あああああ\footnote{いいいいい}うううううううう\par

ここでもゼロ:\PENALTYINFO

%% なんらかのパッケージが \footnote の定義を
%% 書き換えた状況を想定:
\makeatletter
\def\footnote{%\inhibitglue\pltx@foot@penalty\z@
     \@ifnextchar[\@xfootnote{\stepcounter\@mpfn
     \protected@xdef\@thefnmark{\thempfn}%
     \@footnotemark\@footnotetext}}
\makeatother

あああああああああああああああ
あああああ\footnote{いいいいい}うううううううう\par

やっぱりゼロ:\PENALTYINFO

\end{document}
